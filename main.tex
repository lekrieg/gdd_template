\documentclass[12pt, a4paper]{article}
\usepackage[utf8]{inputenc}
\usepackage[brazil]{babel}
\usepackage{tcolorbox}

\title{Game Title}
\author{Author}
\date{MM/DD/Y}

\begin{document}

\maketitle
\newpage
\tableofcontents
\newpage

\section{Description}
A summary of what this game is about, without going into much detail about game mechanics or anything else. After reading the Project Description, it should be clear what type of game you are trying to make (Social, Casual, Hardcore, etc.) and the genre (Puzzle, RPG, FPS, etc.).This section would ideally be one or two paragraphs long. No more than a page for sure\\

For example:

\begin{tcolorbox}
This game design document describes the details for a multi-platform touch based 2D puzzle game with novel mechanics and an original story and characters.\\

The game plays like other match-3 games but introduces some innovations.\\

The name is to be defined but candidates are…
\end{tcolorbox}

\section{Characters}
The reason we start with characters is because you need to introduce them before the Story. If your game doesn’t have Characters and/or Story, you can just jump to the Gameplay section and remove Sections 1 to 3 (or leave them empty).\\

An example of descriptions:
\begin{tcolorbox}
    \subsection{Player}
    \subsubsection*{Character Name}
    Some nice background about the character.
    
    \subsection{NPCs}
    \subsection{Enemies}
\end{tcolorbox}

\section{Story}
Having introduced the characters, it’s a good time to talk about the events that will happen throughout the game.\\

For example:

\begin{tcolorbox}
Gnumies are happily playing inside their castle and causing 
mischief. The Butler is going insane, but everybody is enjoying. 
Joker makes jokes.

German is home watching TV and his mother bothers him. So he 
goes out to spy on the Gnumies. Outside it's raining and German 
is looking envious through the window, getting all wet. 

A strange mysterious person-something gives him a key that he 
can use to enter through a backdoor. He goes in with his army, 
kidnaps and jails female and baby Gnumies, and kicks everybody 
else out of the island…
\end{tcolorbox}

\subsection{Theme}
This is important for when other people read your design. Overall, the theme speaks about what kind of story you want to tell: is it comedy, is it the real life or is it just fantasy.\\

For example:

\begin{tcolorbox}
This is a game about sadness and hardships. There is action and happy moments but between each chapter the story must progress in a way that clearly states that the Gnumies are sad because they lost their home. It must also have a sense of humor and be funny.
\end{tcolorbox}

You can skip this section if you think it’s irrelevant for your game.

\section{Story progression}
So, you have a Story, but how will the game take your players through that story.\\

For example:

\begin{tcolorbox}
The game starts with a short intro scene where the Gnumies
are getting kicked out of their homes. Then they land in an
island and the first chapter begins.\\

The first chapter is the Tutorial. This can be skipped. Here
the levels are few and the Butler introduces the user to 
the mechanics.\\

Once the player beats the tutorial he can advance into the 
First World \textbf{Forest World}.\\

When the player beats the Forest World, he gets the First Key 
and then can choose to open the \textbf{Volcano World} or \textbf{Icy Mountain World}. Once he defeats one of these worlds….\\

It’s very important to develop the world like a place were 
not only this story, but multiple stories could be happening 
at the same time. This opens the door for sequels and 
merchandise.
\end{tcolorbox}

\section{Gameplay}
This is (probably in 99\% of games) the most important section of the GDD. It’s where you describe what your Gameplay (yes, with capital G), will be like.

Since this section can become humongous, we went ahead and divided it in sub-sections that made sense to us. Of course, this is a very subjective topic and what works for us may not work for you.

\subsection{Goals}
In short, why is the player playing your game? It’s good to add this information to a separate section so you don’t have to guess while reading through the whole GDD.\\

For example:
\begin{tcolorbox}
Overall (long term): Help Gnumies return home\\

Gameplay (short term): Defeat the enemies, advance to the next level, etc…
\end{tcolorbox}

\subsection{User Skills}
This is not the most intuitive section, but it really helps to narrow down your scope if you think about what are the skills your player needs to master in order to play your game. Believe us, writing this list will help you find problems in your Game Design. For example, you may be trying to develop a game for kids but realize you require them to do something that is too advanced for their age, or some inputs may be good for Mobile but not for a Console with a Joystick. Also, if your game is going to have Custom HW built around it, then this list will allow you to figure out what components you’ll need to make it work.\\

For example:
\begin{tcolorbox}
    \begin{itemize}
        \item Tap on the screen
        \item Drag and drop
        \item Memory
        \item Puzzle solving
        \item Rearranging pieces
        \item Manage resources
        \item Strategize
    \end{itemize}
\end{tcolorbox}

\subsection{Game Mechanics}
This is where you describe your proper game mechanics. Spare no words, when you circulate this GDD around your team, there has to be the least reasonable amount of doubt about what the gameplay should be like. This is a very good section to add some Artwork or Screenshots of your prototype (we prefer to prototype the mechanics and figure out if they are fun before committing resources to a game).\\

There are complete books and sites with materials about how to describe game mechanics, so we’ll not elaborate with examples here. Linked resources at the bottom.

\subsection{Items and power-ups}
We use this section to elaborate on the Game Mechanics. In order to avoid having a single section with everything in our brains poured into it, we use the section above to describe the core mechanics, and this section to talk about things that can be added to the game in order to improve the fun and empower the player.\\

So, if your game is a match-3 game, then in the previous section you’d go and describe exactly how a match-3 game would work (and adding your variations to the formula).\\

In this section you’d add every power up and item the player can use/encounter/buy and how they would affect the core gameplay.\\

For example:
\begin{tcolorbox}
When finishing a world, you could get a power up related to that world. For example, finishing the volcano world, can give you an item that makes red Gnumies more powerful. It could be a scarf, 
or something they can wear, and those items could be seen in-game later. You can level up items using in-game currency, or use real money to acquire in-game currency packs…. 
\end{tcolorbox}

\subsection{Progression and challenge}
This is also a very subjective section that may or may not work in your design. Our idea behind this section is to elaborate on how the difficulty will increase throughout the game, and making sure we give the player the tools to catch up to it.\\

For example:
\begin{tcolorbox}
Difficulty will advance by making the enemies harder. 
To mitigate difficulty, the user will have to play better, 
level up Gnumies and use items (also level up the items).
\end{tcolorbox}
Also, here we can talk about the way players will unlock new levels or missions.\\

For example:
\begin{tcolorbox}
Each boss drops a key with a jewel of that world’s color. Worlds can be tackled in any order. When the user beats every world and has every key, then he can go and work his way through the last world. The order in which a user tackles each world can be chosen by him. The boss at the end of a world drops a key that can be used to open a different world. Once the item is used, it is lost forever. 
That way, the user must complete the world he selected before opening the next. At that point the difficulty for that world is set
\end{tcolorbox}

\subsection{Losing}
Yes, losing! What are the losing conditions? Time, health, all of them? This is the section where you describe how the player gets to see your "Game Over" screen.\\

For example:
\begin{tcolorbox}
These are the losing conditions: losing by running out of time, losing by running out of moves, losing when there are no available combinations.\\

When the player loses, there must be an image showing the Gnumies wounded/scratched. Maybe they can lose some hair and you can see the skin under the hair.
\end{tcolorbox}

\section{Art Style}
This section is self-explanatory: here’s where you describe your ideas about what the game should look like. Since a picture is worth a thousand words, this is a great place to add some concept art.\\

For example:
\begin{tcolorbox}
This is a 2D isometric game, with high quality 2D sprites. The character design should resemble that of Studio Ghibli.\\

Everything should be very colorful and feel alive, with highly animated scenarios and layered backgrounds….
\end{tcolorbox}

\section{Music and Sounds}
Here is where you describe your Music and Sound FX. Depending on how important this is in your game, then you can split this section in different sub-sections.\\

For example:
\begin{tcolorbox}
The music should have a Retro style, appealing to 8 bit nostalgia but high quality.\\

It’s important that a lot of sound effects praise the user when he does something good. There should be immediate and positive feedback.\\

When time is running low, add a sound that makes the user nervous.\\

The sad scenes should be accompanied by Accordion/Violin music and sound like a sorrowful Tango.\\

For In-Game music, use a more relaxed approach with happy tunes and going up on tempo as the level progresses. When in caves the music should sound muffled.
\end{tcolorbox}

\section{Technical Description}
Here’s where you describe the platforms you’d be launching for and tools you’ll be using or are considering to use throughout your development. This should not be a detailed technical description, for that you have the Technical Design Document (TDD). Here we are just scratching the surface.\\

Example:
\begin{tcolorbox}
Initially, the game will be Mobile Cross-platform:

\begin{itemize}
    \item iOS
    \item Android
    \item Windows Phone
\end{itemize}

Follow with PC standalone version and Facebook Canvas. Could add Mac and/or console support (through e-stores) in a future.\\
Consider the following engines: Unity 3D, Unreal Engine 4.\\
For project management use JIRA.\\
Use Perforce for storing code and assets.\\
TBD properly in Technical Design Document.
\end{tcolorbox}

\section{Marketing \& Funding}
A completely optional section, but write your ideas now so you don’t forget them later. It’s important to think about how you are going to market your game, even before starting your development. It’s also important to know where the money to make the game is coming from.\\

For example:
\begin{tcolorbox}
Prototype the first level, and launch a Kickstarter campaign 
where we show that level.\\

Try to land a publishing deal.\\

Is there any Government funding we can apply to?\\

Create a press kit and send to gaming news websites.\\

Start a YouTube Channel and post development diary 
videos.\\
\end{tcolorbox}

\subsection{Demographics}
It’s important to know who you’ll be targeting, this should spill into the game design. If you are targeting 15 to 25 year old males, then your main character probably shouldn’t be a pink pony (not that there’s anything wrong with that).\\

Example:
\begin{tcolorbox}
Age: 12 to 50\\

Sex: Everyone\\

Casual players mostly\\
\end{tcolorbox}

\subsection{Platforms \& Monetization}
You can add a little more detail about how you are going to approach the release on each platform.\\

For example:
\begin{tcolorbox}
Initially: Free android app with in-game ads, and paid version without ads.\\

Free iOS with ads. Paid iOS version without ads.\\

In game purchases.\\

Consider: Windows 8, Windows Phone 8, XBOX live and Nintendo e-shop.\\
\end{tcolorbox}

\subsection{Localization}
Your supported languages. Just add whatever you have in mind, this is something that probably won’t be a priority until later.\\

Example:
\begin{tcolorbox}
Initially English/Spanish. 
Later update with: Italian, French, German, etc.\\

Consider getting an Asian publisher for expanding to Asia, someone that can help with localization.
\end{tcolorbox}

\section{Ideas to add}
Another completely optional section. If you have ideas that you are not sure if they should go in the game or not, just add them here so you don’t forget them.\\

For example:
\begin{tcolorbox}
    \begin{itemize}
        \item Level designer
        \item Be able to rate levels created by other users
        \item Achievements
        \item Leaderboards
        \item Should the game have a Multiplayer mode?
    \end{itemize}
\end{tcolorbox}

\end{document}
